\documentclass[stu, 12pt, letterpaper, donotrepeattitle, floatsintext, natbib]{apa7}
\usepackage[utf8]{inputenc}
\usepackage{comment}
%\usepackage{marvosym}
\usepackage{graphicx}
\usepackage{float}
%\usepackage[normalem]{ulem}
\usepackage[english,en-tabla]{babel}
\usepackage{amsmath, amsthm, amssymb}
\usepackage{hyperref}
\usepackage{apacite}
\hypersetup{ colorlinks=true,                     %habilitar colorear enlaces
                     linkcolor=black,
                     filecolor=magenta,
                     urlcolor=black,
                     citecolor=black,
                    }
\selectlanguage{english}
%\useunder{\uline}{\ul}{}
%\newcommand{\myparagraph}[1]{\paragraph{#1}\mbox{}\\}
\usepackage[table,xcdraw]{xcolor}
\usepackage{array}
\usepackage{colortbl}
\usepackage{float} % Para utilizar la opción [H]                    



\begin{document} 

\begin{center}
\rule[0.5ex]{\linewidth}{2pt}\vspace*{-\baselineskip}\vspace*{3.2pt}
\rule[0.5ex]{\linewidth}{1pt}\\[\baselineskip]
{\huge UNIVERSIDAD NACIONAL SAN AGUSTIN} \\\vspace{4mm}
{\huge DE AREQUIPA }\\[6mm]
{\Large {FACULTAD DE INGENIERIA DE PRODUCCION Y SERVICIOS}}\\\vspace{3mm}
{\Large {ESCUELA PROFESIONAL DE INGENIERÍA DE SISTEMAS}}\\\vspace{3mm}
\rule[0.5ex]{\linewidth}{1pt}\vspace*{-\baselineskip}\vspace{3.2pt}
\rule[0.5ex]{\linewidth}{2pt}\\
\vspace{2mm}
\includegraphics[scale=0.5]{Imagenes/unsa.png}\\
\vspace{2mm}
{\large \textsc{\textbf{Enfoque de aprendizaje profundo multimodal para la prevención del accidente cerebrovascular isquémico: integración de imágenes y datos clínicos}}\\
\vspace{3mm}
{\large\textsc{ Plan de Tesis}}\\
\vspace{2mm}
{\large\textsc{Susana Milagros Gomez Turpo }}\\
\vspace{2mm}
{\large \textsc{Asesor:}}\\
\vspace{1mm}
{\large\textsc{Dra. Eveling Castro Gutierrez}}\\
\vspace{1mm}
{\large\textsc{Mes 2024}}

\end{center}


\tableofcontents
\newpage
\section{I. PLANTEAMIENTO METODOLÓGICO} \label{cap:CAPI}

\subsection{1.1 Descripción de la Realidad Problemática}

La enfermedad cerebrovascular (ECV) es una de las causas más importantes de mortalidad y discapacidad en el Perú y a nivel mundial (\cite{MALAGA2018}), si bien es cierto que la ECV son un conjunto de trastornos que afectan los vasos sanguíneos del cerebro y son responsables de un porcentaje significativo de muertes prematuras  a nivel global y local (\cite{bernaberevista}), en términos de prevención el accidente cerebrovascular (ACV) es el más estudiado y focalizado.
\newline

Según la Organización Mundial de la Salud, los ACV son la segunda causa de muerte a nivel mundial (\cite{incn2022}), cobrando más de 6.6 millones de vidas cada año. En el Perú es una de las principales causas de muerte prematura y discapacidad. También se observa una muy importante asociación con factores de riesgos modificables como la hipertensión arterial y la diabetes con esta patología. Además, el ictus es una enfermedad de alto costo y comúnmente es asumido por la familia, en especial cuando el paciente es dado de alta de los hospitales (\cite{incn2024}).Controlar los factores de riesgo puede reducir significativamente la incidencia de ACV (\cite{acv_s_f}).
\newline

Carlos Abanto, jefe del Centro de Investigación en Enfermedad Cerebrovascular del Instituto Nacional de Ciencias Neurológicas (INCN), abordó la prevención primaria en enfermedades cerebrovasculares. Destacó que estas enfermedades son la segunda causa de muerte a nivel mundial, solo superadas por el Alzheimer, y una importante causa de discapacidad.
\newline


Perú está en una fase de crecimiento económico y experimenta un cambio en su perfil epidemiológico: disminución de enfermedades infecciosas y aumento de enfermedades no comunicables. Un estudio de 2007 reveló que el 10-11\% de los pacientes muere a los 3 meses después del alta, y esta cifra puede estar subestimada debido a la falta de seguimiento.
\newline


El mismo estudio de 2007 mostró que aproximadamente la mitad de los pacientes mantienen independencia funcional después de un ictus, mientras que la otra mitad requiere asistencia. La edad es un factor crucial en el pronóstico y la severidad del evento.
\newline

Por otro lado, los avances tecnológicos en medicina, como las tomografías computarizadas sin contraste, resonancias magnéticas (MRI) y análisis clínico, han revolucionado el diagnóstico y tratamiento de diversas enfermedades. Sin embargo, en Perú, la limitada disponibilidad de resonancia magnética resalta la necesidad de desarrollar herramientas diagnósticas alternativas, especialmente para abordar el ACV. En este contexto, un modelo de aprendizaje profundo multimodal adaptado a imágenes de TC podría ofrecer una solución eficaz. Este enfoque no solo podría acelerar el proceso de diagnóstico, sino también mejorar los resultados clínicos, especialmente en situaciones de emergencia.
\newline

\subsection{1.2 Problema Principal}
En Perú, la prevención de los accidentes cerebrovasculares (ACV) es crucial debido a su impacto significativo en la salud pública y económica. Los ACV son una de las principales causas de muerte y discapacidad en el país, lo que subraya la importancia de implementar estrategias efectivas para reducir su incidencia y mejorar los resultados de salud de la población (\cite{MALAGA2018}).
\newline

Las estrategias de prevención primaria se centran en modificar los factores de riesgo conocidos, que incluyen la hipertensión, la diabetes, la dislipidemia (colesterol alto), el tabaquismo, la obesidad, la inactividad física y una dieta poco saludable. Promover estilos de vida saludables, como una dieta equilibrada y la actividad física regular, juega un papel crucial en la reducción del riesgo de ACV.
\newline

A pesar de los avances en tecnología de imágenes médicas, la variabilidad en la interpretación de los resultados y la falta de disponibilidad de equipos avanzados en muchas regiones, incluido Perú, plantean desafíos significativos. Aquí es donde se hace uso del aprendizaje profundo multimodal, una subdisciplina del aprendizaje automático que ha demostrado un rendimiento sobresaliente en tareas de reconocimiento y segmentación de imágenes. El uso de técnicas de aprendizaje profundo multimodal podría ayudar a mejorar la precisión en el diagnóstico temprano y la gestión de factores de riesgo, incluso en entornos con recursos limitados (\cite{incn2024}).
\newline

\subsection{1.3 Objetivos de la Investigación}

\subsubsection{1.3.1 Objetivo General}
Proponer un modelo de aprendizaje profundo multimodal basado en deep learning para ayudar al diagnóstico de ACV a partir de tomografías computarizadas sin contraste  o resonancias magnéticas e historias clínicas.

\subsubsection{1.3.2 Objetivos Específicos}
\begin{APAenumerate}
    \item Realizar una revisión del estado del arte sobre el uso de los modelos de aprendizaje profundo multimodal en el diagnóstico temprano de ACV, identificando los enfoques, desafíos y avances más recientes.
     \item Seleccionar un modelo de aprendizaje profundo multimodal adecuado para apoyo al diagnóstico temprano de ACV a partir de tomografías computarizadas sin contraste o resonancias magnéticas e historias clínicas , basado en su rendimiento en estudios previos.    
     \item Diseñar experimentos para evaluar el rendimiento del modelo seleccionado utilizando conjuntos de datos y métricas de evaluación apropiadas.
    \item Evaluar resultados experimentales del modelo elegido.

\end{APAenumerate}
  
\subsection{1.4 Hipótesis de la Investigación}
\begin{APAenumerate}
La integración de un modelo de aprendizaje profundo multimodal que combine datos de tomografías computarizadas sin contraste, resonancias magnéticas y datos clínicos históricos proporcionará un apoyo significativo en el diagnóstico temprano de accidentes cerebrovasculares (ACV) isquémicos. Se espera que este enfoque aumente la precisión y eficiencia del diagnóstico en comparación con los métodos tradicionales, permitiendo una intervención médica más oportuna y eficaz. 
\end{APAenumerate}

\subsection{1.5 Variables e Indicadores}
\subsubsection{1.5.1 Variable Independiente}
La variable independiente en esta investigación es el modelo de aprendizaje profundo multimodal. Este modelo combina datos de diferentes fuentes, específicamente tomografías computarizadas sin contraste, resonancias magnéticas y datos clínicos históricos, para apoyar el diagnóstico temprano de accidentes cerebrovasculares (ACV) isquémicos.
\begin{table}[h!]
    \centering
    \begin{tabular}{|c|c|c|}
        \hline
        \rowcolor[HTML]{C0C0C0}
        \textbf{Indicadores} & \textbf{Índices}  \\
        \hline
        Precisión de diagnóstico & 91\%\\
        \hline
        Sensibilidad (Recall/Sensitivity)&  88\%\\
        \hline
        Especificidad (Specificity)&  93\%\\
        \hline
    \end{tabular}
    \caption{Indicadores e Índices del Modelo Multimodal para Diagnóstico de ACV}
    \label{tab:preciseg}
\end{table}
\subsubsection{1.5.2 Variable Dependiente}
\begin{itemize}
    \item \textbf{Precisión del Diagnóstico}: La exactitud del modelo de aprendizaje profundo multimodal para identificar correctamente los casos de ACV isquémico en comparación con los métodos diagnósticos tradicionales.
    \item \textbf{Eficiencia del Diagnóstico}: El tiempo requerido para realizar un diagnóstico preciso utilizando el modelo de aprendizaje profundo multimodal.
\end{itemize}
\begin{table}[h!]
    \centering
    \begin{tabular}{|c|c|}
        \hline
         \rowcolor[HTML]{C0C0C0}
        \textbf{Indicadores} & \textbf{Índices} \\ \hline
        Precisión del Diagnóstico &  85\% y 95\%,  \\ \hline
        Eficiencia del Diagnóstico & 90\% y 95\%,  \\ \hline
    \end{tabular}
    \caption{Indicadores e Índices del Modelo Multimodal para Diagnóstico de ACV}
    \label{table:indicadores_indices}
\end{table}

\subsection{1.6 Viabilidad de la Investigación}
\subsubsection{1.6.1 Viabilidad técnica}

\begin{itemize}
    \item Posibilidad tecnológica: para el desarrollo del proyecto se requiere una computadora y software especializado. Asimismo, se requieren imágenes de TC y historias clínicas. Se cuenta con los equipos e imágenes de TC.
    \item Infraestructura: Los laboratorios de la escuela profesional de ingeniería de sistemas; los cuales son suficientes para los equipos requeridos. 
\end{itemize}
En base a lo anterior se puede concluir que es viable técnicamente.
\subsubsection{1.6.2 Viabilidad operativa}
Se cuenta con el compromiso tanto del investigador, asesor de tesis como del médico especialista en neurología para llevar a cabo la investigación. Asimismo, al tener el soporte de un médico especialista para el desarrollo de la investigación, se puede contar con una institución médica para validar los resultados. Por lo tanto, se puede determinar que es factible operacionalmente.
\subsubsection{1.6.3 Viabilidad económica}
El plan de tesis presentado se autofinanciará.
\subsection{1.7 Justificación e Importancia de la Investigación} 
\subsubsection{1.7.1 Justificación}
La detección temprana y la intervención preventiva son pilares fundamentales en la gestión del riesgo de accidente cerebrovascular (\cite{MALAGA2018}). Los sistemas multimodales que integran imágenes médicas avanzadas y datos clínicos detallados, utilizando técnicas de aprendizaje profundo multimodal, ofrecen una estrategia efectiva para abordar este desafío de manera proactiva.
\newline

Las imágenes médicas, como las resonancias magnéticas y las tomografías computarizadas sin contraste, proporcionan información detallada sobre la anatomía cerebral y la salud vascular. Estos datos visuales pueden revelar signos precursores de riesgo, como placas arteriales, estenosis o microhemorragias, que son indicativos de vulnerabilidad vascular (\cite{HUERFANO2016}). Combinados con datos clínicos que incluyen factores de riesgo tradicionales como la presión arterial, el colesterol, el historial médico y los hábitos de vida, estos sistemas multimodales permiten una evaluación integral y precisa del riesgo individual de ACV (\cite{BORSOS2024102719}).
\newline

La capacidad de aprendizaje profundo multimodal para aprender representaciones complejas y no lineales a partir de datos heterogéneos es crucial en este contexto (\cite{SUN2023104482}). Los algoritmos de aprendizaje profundo multimodal pueden identificar patrones sutiles y relaciones entre diferentes tipos de datos, lo que mejora significativamente la capacidad predictiva del sistema. Esto no solo facilita la identificación temprana de individuos en riesgo elevado, sino que también permite la personalización de estrategias preventivas específicas para cada paciente (\cite{ngiam2011multimodal, ramachandram2017deep, esteva2019guide, litjens2017survey}).
\newline

Desde una perspectiva preventiva, estos sistemas multimodales pueden recomendar intervenciones proactivas, como cambios en el estilo de vida, terapias farmacológicas preventivas o programas de monitoreo regular. Estas medidas no solo pueden reducir el riesgo de un evento cerebrovascular agudo, sino que también pueden mejorar la calidad de vida a largo plazo al prevenir daños cerebrales potencialmente devastadores (\cite{esteva2019guide, litjens2017survey}).
\newline

Además de su aplicación clínica directa, estos sistemas también ofrecen oportunidades significativas para la investigación médica. La capacidad de analizar grandes volúmenes de datos longitudinales puede revelar nuevos biomarcadores predictivos y mejorar la comprensión de los mecanismos subyacentes del riesgo de ACV. Esto puede abrir nuevas vías para el desarrollo de intervenciones más efectivas y personalizadas en la prevención del ACV (\cite{esteva2019guide, litjens2017survey}).

\newline
En conclusión, los sistemas de aprendizaje profundo multimodales representan una herramienta poderosa y prometedora en la prevención del riesgo de accidente cerebrovascular. Al integrar datos complejos de múltiples fuentes, estos sistemas no solo mejoran la capacidad diagnóstica y predictiva, sino que también ofrecen la posibilidad de transformar la atención médica preventiva al identificar y abordar el riesgo de ACV antes de que se produzcan eventos clínicamente significativos.

\subsubsection{1.7.2 Importancia}

La prevención del accidente cerebrovascular (ACV) es un desafío médico significativo debido a su potencial impacto devastador en la salud y la calidad de vida de los individuos. En este contexto, los sistemas  que integran imágenes médicas avanzadas y datos clínicos mediante técnicas de aprendizaje profundo multimodal emergen como herramientas cruciales para mejorar la detección temprana, la evaluación de riesgos y la implementación de estrategias preventivas efectivas (\cite{esteva2019guide, litjens2017survey}).

\begin{itemize}
\item Mejora en la Precisión Diagnóstica y Evaluación de Riesgos
Los sistemas de aprendizaje profundo multimodales permiten una evaluación integral del riesgo de ACV al combinar datos detallados de múltiples fuentes. Las imágenes médicas proporcionan una visión directa de la anatomía cerebral y la salud vascular, identificando anomalías sutiles que podrían indicar predisposición al ACV, como placas arteriales o microhemorragias. Estos datos visuales se complementan con información clínica que incluye factores de riesgo tradicionales como la presión arterial, el colesterol y el historial médico. La capacidad de los algoritmos de aprendizaje profundo multimodal para aprender y correlacionar estos datos mejora significativamente la precisión diagnóstica y la evaluación de riesgos, permitiendo la identificación temprana de individuos en riesgo elevado (\cite{esteva2019guide, litjens2017survey}).


\item Los sistemas multimodales generan perfiles de riesgo personalizados basados en datos biomédicos y de estilo de vida. Estas estrategias no solo reducen el riesgo de ACV agudo, sino que también mejoran la gestión a largo plazo de la salud vascular y cerebral del paciente, promoviendo una mejor calidad de vida y mitigando la carga económica asociada (\cite{esteva2019guide, litjens2017survey}).


\item Los sistemas de aprendizaje profundo multimodal no solo tienen aplicaciones clínicas directas, sino que también permiten analizar grandes volúmenes de datos para revelar biomarcadores predictivos, mejorar la comprensión del riesgo de ACV y avanzar en la fisiopatología y terapias.
(\cite{esteva2019guide, litjens2017survey}).
\end{itemize}

En resumen, los sistemas de aprendizaje profundo multimodal representan una herramienta integral y prometedora en la prevención del riesgo de accidente cerebrovascular. Al integrar datos complejos de múltiples fuentes, estos sistemas mejoran la precisión diagnóstica, personalizan las estrategias preventivas y promueven la investigación médica, transformando así la gestión de la salud vascular y cerebral para una población cada vez más susceptible a este tipo de eventos.
\subsubsection{1.7.3 Alcance}

El alcance de un estudio o proyecto que utilice sistemas de aprendizaje profundo multimodal para la prevención del riesgo de accidente cerebrovascular (ACV) puede abarcar varios aspectos clave:
\begin{itemize}
    \item Desarrollo y Validación de Modelos Predictivos: El estudio se enfocará en desarrollar y validar modelos de aprendizaje profundo multimodal que integren imágenes médicas y datos clínicos para predecir el riesgo individual de ACV. Esto incluye la selección adecuada de algoritmos de aprendizaje profundo multimodal, la optimización de hiperparámetros y la evaluación exhaustiva de la precisión y la robustez del modelo.

    \item Integración de Datos Multimodales: Es fundamental integrar y procesar datos heterogéneos de manera efectiva. El alcance podría incluir la implementación de técnicas avanzadas de procesamiento de imágenes (por ejemplo, redes neuronales convolucionales) y procesamiento de lenguaje natural para manejar datos textuales clínicos, asegurando una representación completa y precisa del estado de salud del paciente.

    \item Personalización del Tratamiento Preventivo: El estudio explorará cómo los modelos desarrollados pueden ser utilizados para personalizar estrategias preventivas. Esto implica identificar factores de riesgo específicos y recomendar intervenciones adaptadas a las necesidades individuales de cada paciente, como cambios en el estilo de vida, terapias farmacológicas preventivas y programas de seguimiento médico.

    \item Validación Clínica y Aplicabilidad Práctica: Es esencial validar la eficacia clínica de los modelos en entornos reales de atención médica. El alcance incluirá la colaboración con profesionales de la salud para implementar y evaluar la utilidad práctica de los sistemas multimodales en la detección temprana y la gestión del riesgo de ACV.

    \item Investigación Continua y Mejora del Modelo: Además, el estudio contemplará la investigación continua para mejorar los modelos de aprendizaje profundo multimodal, incorporando nuevos datos y conocimientos médicos para mantener la precisión predictiva a lo largo del tiempo. Esto podría incluir la identificación de nuevos biomarcadores, la optimización de algoritmos y la adaptación a cambios en las prácticas clínicas y epidemiológicas.
\end{itemize}

\subsection{1.8 Línea, Tipo y Nivel de Investigación}
\subsubsection{1.8.1 Línea de la investigación}
Inteligencia Artificial, sublínea de Aprendizaje Profundo Multimodal basado en Deep Learning.
\subsubsection{1.8.2 Tipo de la investigación}
El tipo de investigación planteada es investigación aplicada. Considerando que la investigación aplicada tiene diversos tipos, la investigación propuesta será experimental, ya que para el desarrollo de la investigación es fundamental analizar las imágenes de TC e historias clínicas controladas por un médico especialista en el área de neurología.
\subsubsection{1.8.3 Nivel de la investigación}
El nivel de la investigación es explicativo y aplicado.
\begin{itemize}
    \item Explicativo: Se busca no solo describir el fenómeno del diagnóstico de ACV mediante modelos de aprendizaje profundo multimodal, sino también entender y explicar cómo la integración de datos de tomografías computarizadas sin contraste,resonancias magnéticas y datos clínicos históricos puede mejorar la precisión y eficiencia del diagnóstico temprano de ACV isquémicos.
    \item Aplicado: La investigación tiene un enfoque práctico con el objetivo de desarrollar y evaluar un modelo específico que pueda ser utilizado en entornos clínicos reales. La meta es proporcionar una herramienta que mejore los resultados clínicos y la eficiencia del diagnóstico de ACV isquémicos, contribuyendo directamente a la práctica médica y la salud pública.
\end{itemize}

\subsubsection{1.8.4 Diseño de la investigación}
El diseño de la investigación planteada es experimental o de laboratorio. Inicialmente, se utilizarán imágenes médicas (como tomografías computarizadas sin contraste) y datos de historias clínicas para desarrollar un modelo de aprendizaje profundo multimodal. Se considerará la colaboración con un especialista en neurología en el futuro para validar clínicamente los resultados obtenidos. Esto permitirá asegurar la precisión y efectividad del modelo en el diagnóstico temprano de ACV isquémicos, integrando datos complejos de manera robusta y validada desde el punto de vista médico.

\begin{figure}[!h]
	\centering
	\includegraphics[width=16cm]{PlanTesis_FormatoUNSA_2024/Imagenes/diagrama1.png}
	\caption{Esquema de las etapas a seguir para el desarrollo de la investigación. La primera etapa se encarga del preprocesamiento de data. La segunda etapa tiene por finalidad la extracción de características. La tercera etapa se encarga de la fusión de los datos para finalmente dar un diagnóstico .Fuente: Elaboración Propia.}
	\label{Figure:figure3}
\end{figure}


\subsubsection{1.8.5 Técnicas}
\begin{itemize}

\item Revisión sistemática de la literatura concerniente a técnicas de aprendizaje profundo para mejorar la precisión en el diagnóstico, integrando de manera efectiva los datos obtenidos.
\item La presente investigación emplea una metodología basada en la observación experimental para analizar imágenes de TC sin contraste y datos clínicos. A partir de las imágenes médicas e historias clínicas, se extraerá la información relevante bajo condiciones controladas. 


\end{itemize}
\subsubsection{1.8.6 Instrumentos}


\begin{itemize}
    \item \textbf{Imágenes de Tomografía Computarizada (TC)}
    \begin{itemize}
       Obtención de imágenes detalladas del cerebro para detectar y analizar accidentes cerebrovasculares.
    \end{itemize}
    
    \item \textbf{Historias Clínicas}
    \begin{itemize}
        Utilización de historias clínicas para obtener datos como edad, sexo y algunas enfermedades de los pacientes.
    \end{itemize}
    
    \item \textbf{Técnicas de Análisis de Datos}
    \begin{itemize}
        \item \textbf{Software de Procesamiento de Imágenes Médicas:} Programas para la visualización y análisis de imágenes de TC.
        \item \textbf{Herramientas de Análisis Clínico:} Aplicaciones para realizar análisis estadísticos y clínicos de los datos.
    \end{itemize}
    
    \item \textbf{Técnicas de Aprendizaje Profundo}
    \begin{itemize}
        \item \textbf{Librerías de Aprendizaje Profundo:} TensorFlow, Keras o PyTorch para desarrollar y evaluar modelos.
        \item \textbf{Hardware de Alto Rendimiento:} GPUs y TPUs para acelerar el entrenamiento y la ejecución de modelos.
    \end{itemize}
\end{itemize}


\subsubsection{1.8.7 Cronograma}


\begin{table}[H]
\centering
\begin{tabular}{|>{\centering\arraybackslash}m{5cm}|>{\centering\arraybackslash}m{5cm}|>{\centering\arraybackslash}m{3cm}|>{\centering\arraybackslash}m{3cm}|}
\hline
\rowcolor[HTML]{C0C0C0} 
\textbf{ACTIVIDAD} & \textbf{DESCRIPCIÓN DE LA ACTIVIDAD} & \textbf{FECHA DE INICIO} & \textbf{FECHA DE FIN} \\ \hline
 Revisión del estado del arte  & Identificar enfoques, desafíos y avances más recientes. & 01/04/2024  & 18/06/2024 \\ \hline
Seleccionar un modelo de aprendizaje profundo multimodal&  Seleccionar un modelo basado en su rendimiento en estudios previos & 25/04/2024  & 09/09/2024 \\ \hline
Diseño de experimentos & Evaluación de el rendimiento del modelo seleccionado & 20/09/2024  & 17/10/2024   \\ \hline
Evaluar resultados & Evaluación de resultados del modelo seleccionado & 20/10/2024  & 17/11/2024   \\ \hline
\end{tabular}
\caption{Cronograma de Actividades}
\end{table}


\subsubsection{1.8.8 Temario del Informe Final}
\begin{itemize}

\item[I.] PLANTEAMIENTO METODOLÓGICO
\begin{itemize}
\item [1.1] Descripción de la Realidad Problemática 
\item [1.2] Problema Principal 
\item [1.3] Objetivos de la Investigación 
\item [1.4] Hipótesis de la Investigación 
\item [1.5] Variables e Indicadores
\item [1.6] Viabilidad de la Investigación 
\item [1.7] Justificación e Importancia de la Investigación
\item [1.8] Linea, Tipo y Nivel de Investigación

\end{itemize}
\item[II.] MARCO TEORICO
\begin{itemize}
\item [2.1] Antecedentes de investigación 
\item [2.2] Estado del Arte
\item [2.3] Marco Conceptual
\end{itemize}
\item[III.] DESARROLLO DE LA PROPUESTA
\item[IV.] DESARROLLO DE PRUEBAS CON EL USO DE METRICAS
\item[V.] EXPERIMENTACION Y VALIDACION
\end{itemize}
CONCLUSIONES 

RECOMENDACIONES

RABAJOS FUTUROS

REFERENCIAS BIBLIOGRÁFICAS

ANEXOS


 %Inserta el capítulo 1

\newpage
\section{II. MARCO TEORICO.} \label{cap:CAPII}

\subsection{2.1 Antecedentes de investigación}

La capacidad humana de percibir y comprender el mundo se basa en cinco sentidos básicos: oído, tacto, olfato, gusto y vista. Al poseer estas cinco modalidades, somos capaces de percibir y comprender el mundo que nos rodea. Así, multimodal significa combinar diferentes canales de información simultáneamente para comprender nuestro entorno (\cite{akkus2023multimodaldeeplearning}). 
\newline

La investigación en el manejo del riesgo de accidente cerebrovascular (ACV) ha evolucionado significativamente con el avance de los sistemas multimodales que integran imágenes médicas y datos clínicos (\cite{Cui_2023}). Estos sistemas han permitido una mejor comprensión de los factores de riesgo y la predicción temprana de eventos cerebrovasculares mediante el uso de modelos de aprendizaje profundo (\cite{9985596}).
\newline

Anteriormente, los métodos tradicionales se centraban en la evaluación clínica y el uso de biomarcadores simples. Sin embargo, con el advenimiento del aprendizaje profundo, se ha logrado una integración más compleja de datos, lo que ha mejorado la precisión diagnóstica y la capacidad predictiva del ACV (\cite{CuiLStroke}).
\newline

Estudios previos han demostrado que la combinación de imágenes como tomografías computarizadas con datos clínicos permite una personalización más efectiva de las estrategias preventivas y terapéuticas . Esto es crucial dado que el ACV es una de las principales causas de morbilidad y mortalidad a nivel mundial, destacando la necesidad urgente de métodos diagnósticos y terapéuticos más avanzados (\cite{s21020460,app10196791})
\newline

En resumen, los antecedentes de investigación destacan la transición hacia sistemas multimodales con aprendizaje profundo como un avance significativo en la gestión del riesgo de ACV, promoviendo una mejor atención personalizada y una reducción de la carga global de la enfermedad.

\subsection{2.2 Estado del Arte } 

En el ámbito médico, los datos se clasifican generalmente en tres categorías principales: datos de imágenes, datos clínicos y datos ómicos. Cada modalidad representa información importante, la combinación de diferentes modalidades proporciona una visión más completa de la enfermedad (\cite{BEHRAD2022117006,Hel2021stroke}). 
\newline

Estudios recientes han demostrado que los modelos multimodales que combinan datos de imágenes y clínicos, como antecedentes médicos, edad, sexo y tratamientos médicos, pueden predecir y reducir el riesgo de ACV con mayor precisión, ayudando a los médicos a comprender mejor las características de los pacientes y la evolución de la enfermedad (\cite{Hel2021stroke,app10196791}). Por ejemplo, uno de los estudios mostró que la combinación de imágenes de MRI con datos clínicos mejoró significativamente la capacidad de predecir eventos cerebrovasculares (\cite{Fang2021stroke,BEHRAD2022117006,Gkantzios2023stroke}).
\newline

Es así que los sistemas multimodales con \textit{aprendizaje profundo} para la predicción y manejo del riesgo de ACV, las TC sin contraste pueden ser procesadas mediante técnicas de procesamiento de imágenes y análisis de patrones para extraer características relevantes. Estas características se pueden combinar con datos clínicos en un modelo predictivo para mejorar la precisión y la personalización del manejo del paciente.
\newline

En este contexto, los modelos de aprendizaje profundo multimodales permiten la extracción de características complejas y la identificación de patrones que no son evidentes con métodos tradicionales. Modelos como las redes neuronales convolucionales (CNN) y las redes neuronales recurrentes (RNN) se utilizan para analizar imágenes y datos clínicos respectivamente, mientras que arquitecturas como las redes neuronales profundas (DNN) pueden combinar estas modalidades para mejorar la precisión de las predicciones (\cite{app10196791,Chahine2023}).
\newline

A pesar de los avances, la integración efectiva de datos multimodales enfrenta varios desafíos. La heterogeneidad de los datos, la necesidad de grandes volúmenes de datos etiquetados para entrenar modelos, y la validación clínica robusta en diferentes contextos médicos y poblacionales son algunos de los obstáculos a superar (\cite{Chahine2023}).
\newline

Sin embargo, la continua evolución de las técnicas de aprendizaje profundo y la creciente disponibilidad de datos clínicos y de imágenes de alta calidad prometen avances significativos en la prevención y tratamiento del ACV. La colaboración interdisciplinaria y el desarrollo de estándares para la integración de datos multimodales serán cruciales para el éxito de estos enfoques.

\subsection{2.3 Marco Conceptual} 
\subsubsection{2.3.1 Conceptos Biológicos} 
\paragraph{Anatomía del Cerebro}
El cerebro se divide en tres principales componentes:
\begin{itemize}
    \item \textbf{Cerebro}: Compuesto por los hemisferios derecho e izquierdo, el cerebro controla el inicio y la coordinación del movimiento, la temperatura, el tacto, la visión, la audición, el juicio, el razonamiento, la resolución de problemas, las emociones y el aprendizaje.
    
    \item \textbf{Tronco Encefálico}: Incluye el mesencéfalo, la protuberancia y el bulbo raquídeo. Regula funciones esenciales como el movimiento de los ojos y la boca, la transmisión de sensaciones, el hambre, la respiración, la conciencia, la función cardíaca, la temperatura corporal, y movimientos involuntarios como estornudos y vómitos.
    
    \item \textbf{Cerebelo}: Ubicado en la parte posterior de la cabeza, coordina los movimientos musculares voluntarios y mantiene la postura, el equilibrio y el balance.

    \item \textbf{Puente de Varolio}: En el tronco encefálico, controla movimientos oculares y faciales, así como la sensibilidad facial y el equilibrio.
    \item \textbf{Bulbo Raquídeo}: La parte más baja del tronco encefálico, controla funciones vitales como el ritmo cardíaco y la respiración.
    \item \textbf{Médula Espinal}: Extiende desde la base del cerebro hasta la parte inferior de la espalda, transmitiendo mensajes entre el cerebro y el resto del cuerpo.
\end{itemize}

\subparagraph{Lóbulos del Cerebro}

\begin{itemize}
    \item \textbf{Lóbulo Frontal}: Ubicado en la parte frontal, se encarga de la personalidad y el movimiento.
    \item \textbf{Lóbulo Parietal}: En la parte media, ayuda a identificar objetos, comprender relaciones espaciales y interpretar el dolor y el tacto.
    \item \textbf{Lóbulo Occipital}: En la parte posterior, está relacionado con la visión.
    \item \textbf{Lóbulo Temporal}: En los lados del cerebro, está asociado con la memoria, el habla y el sentido del olfato.
\end{itemize}
\cite{hopkins_ct_brain}
\begin{figure}[!h]
	\centering
	\includegraphics[width=14cm]{PlanTesis_FormatoUNSA_2024/Imagenes/basic-anatomy-of-the-brain_spanish.jpg}
    \caption{Estructura del cerebro.}
    \label{fig:estructura_cerebro}
\end{figure}



\subsubsection{2.3.2 Tomografía Computarizada (TC)}

La tomografía computarizada (TC) del cerebro es un procedimiento no invasivo que utiliza rayos X para crear imágenes horizontales detalladas del cerebro, llamadas cortes, ofreciendo información más precisa sobre el tejido cerebral y las estructuras internas en comparación con las radiografías estándar (\cite{hopkins_ct_brain}) . La TC es la modalidad de imagen primaria utilizada en la mayoría de los centros para evaluar a pacientes con síntomas de accidente cerebrovascular agudo, debido a su amplia disponibilidad, rapidez y seguridad (\cite{peralta_agudelo_2023}).
\newline
Durante la TC, el haz de rayos X gira alrededor del cuerpo, permitiendo múltiples vistas del cerebro. Los datos de los rayos X son procesados por una computadora y mostrados en una imagen bidimensional (2D).
\newline
La TC del cerebro puede realizarse con o sin el uso de un medio de contraste, una sustancia que mejora la claridad de las imágenes. En algunos casos, el paciente debe ayunar antes del examen si se usa contraste
(\cite{hopkins_ct_brain}).

\subsubsection{2.3.3 Tomografía computarizada (TC) sin contraste}
Además de la información clínica, que describe el estado del paciente en su conjunto, los médicos se centran principalmente en las técnicas de diagnóstico por imagen para comprender mejor la situación . Tradicionalmente, el Consejo de Accidentes Cerebrovasculares de la Asociación Estadounidense del Corazón recomienda la tomografía computarizada sin contraste (NCCT) como la primera modalidad de elección para la investigación de accidentes (\cite{BORSOS2024102719}), debido a las siguientes características  (\cite{peralta_agudelo_2023}):
\begin{itemize}
    \item \textbf{Visualización rápida}: Las TC proporcionan imágenes detalladas del cerebro rápidamente, lo cual es esencial en situaciones de emergencia, como la evaluación de un accidente cerebrovascular (ACV) o traumatismo craneoencefálico .
    \item \textbf{Diferenciación de tejidos}: Pueden diferenciar entre diferentes densidades de tejido, permitiendo la identificación de estructuras anatómicas y patologías como hemorragias, tumores y edema cerebral.
    \item \textbf{No invasivas}: No requieren la inyección de medios de contraste, lo que elimina el riesgo de reacciones adversas a estos productos .
\end{itemize}


\paragraph{Limitaciones}
\begin{itemize}
\item \textbf{Menor sensibilidad para ciertas lesiones}: Las TC sin contraste pueden no detectar lesiones isquémicas pequeñas o cambios sutiles en etapas muy tempranas.
\item \textbf{Exposición a radiación}: Aunque la dosis de radiación es moderada, es importante considerar la exposición acumulativa en pacientes que requieren múltiples estudios 
\end{itemize}

\subsubsection{2.3.4 Accidentes Cerebrovasculares}
Un accidente cerebrovascular es una emergencia médica que requiere atención inmediata. Las lesiones cerebrales y otras consecuencias se pueden evitar si se actúa de forma temprana. Existen dos tipos principales de accidente cerebrovascular (\cite{upadhyay2022}) :
\begin{itemize}
    \item \textbf{Accidente cerebrovascular hemorrágico}: Cuando se rompe un vaso sanguíneo, se produce un accidente cerebrovascular hemorrágico. Los aneurismas o las malformaciones arteriovenosas (MAV) son las causas más comunes de accidente cerebrovascular hemorrágico.
    \item \textbf{Accidente cerebrovascular isquémico}: Cuando el flujo sanguíneo a una parte del cerebro se bloquea o disminuye, el tejido cerebral se ve privado de oxígeno y nutrientes, lo que da lugar a un accidente cerebrovascular isquémico.
\end{itemize}



\subsubsection{2.3.5 Factores de riesgo}

Los factores de riesgo importantes en la prevención del accidente cerebrovascular se pueden dividir en factores de riesgo modificables y no modificables (\cite{feiginstroke}).

\paragraph{Factores de Riesgo No Modificables:}
    \begin{itemize}
        \item \textbf{Edad: } El riesgo aumenta con la edad, siendo del 0.5\% en personas de 18 a 44 años y del 11.2\% en mayores de 75 años.
        \item \textbf{Sexo: } Mayor prevalencia en hombres, con diferencias que disminuyen después de los 50 años debido a la menopausia en mujeres.
        \item \textbf{Raza: } Mayor riesgo en personas de raza negra, debido a una mayor prevalencia de hipertensión.
        \item \textbf{Bajo Peso al Nacer: } Contribuye al riesgo de enfermedad vascular.
        \item \textbf{Historia Familiar: } Un familiar con enfermedad cerebrovascular incrementa el riesgo. El riesgo es 1.4 veces mayor si el familiar es materno y 2.4 veces mayor si es paterno.
    \end{itemize}

\paragraph{Factores de Riesgo Modificables}
\begin{itemize}
    \item \textbf{Tabaquismo: } Aumenta el riesgo de ictus isquémico y hemorrágico, con un riesgo relativo 1.9 veces mayor en fumadores.
    \item \textbf{Hipertensión Arterial: } Principal factor de riesgo, especialmente en combinación con la edad avanzada. El riesgo relativo puede alcanzar hasta 8.
    \item \textbf{Diabetes: } Incrementa el riesgo con una prevalencia del 7.3\%, con un riesgo relativo de 1.8 para diabéticos.
    \item \textbf{Colesterol Elevado: } Mantener el LDL por debajo de 100 mg/dL es crucial en pacientes con factores de riesgo.
    \item \textbf{Fibrilación Auricular No Valvular: } Su manejo ha mejorado con nuevos anticoagulantes, aunque el costo sigue siendo una preocupación.
\end{itemize}

\paragraph{Otros Factores:Consideración de datos adicionales}
\newline
La combinación de hipertensión, diabetes y tabaquismo multiplica el riesgo de eventos cerebrovasculares. Los programas de intervención deben enfocarse en controlar la hipertensión y la diabetes, promover cambios en el estilo de vida, y usar medicamentos para el manejo del colesterol y la fibrilación auricular.

\begin{itemize}
    \item \textbf{Diagnóstico Mejorado: } La fibrilación auricular (FA) se diagnostica más frecuentemente con el uso de Holter prolongados y dispositivos de monitoreo a largo plazo.
    \item \textbf{Tratamiento: } Los nuevos anticoagulantes son efectivos en la prevención de eventos cerebrovasculares con menos interacciones y sin problemas relacionados con el consumo de vegetales verdes. La evaluación del riesgo debe utilizar escalas como CHADS2 y CHA2DS2-VASc.
\end{itemize}


\subsubsection{2.3.6 Prevención de accidentes cerebrovasculares}

La prevención es una serie de acciones que se hacen para que no ocurra en el evento. Se divide en:

\paragraph{Prevención Primaria}
El objetivo es evitar el primer evento cerebrovascular.
\begin{itemize}
    \item \textbf{Acciones Clave: } Identificación temprana de factores de riesgo, promoción de un estilo de vida saludable, y detección precoz mediante screening de grupos de riesgo \cite{incn2018}. Para ello es importante realizar una evaluación inicial que comtemple la medición de presión arterial, glucosa en ayunas y colesterol, así como la historia familiar de enfermedad cerebrovascular.

    \item \textbf{El uso de herramientas de Evaluación: } El estudio de Framingham y la calculadora de riesgo de la American Heart Association ayudan a estimar el riesgo basado en factores como presión arterial, diabetes y colesterol.
\end{itemize}

\paragraph{Prevención Secundaria}
\begin{itemize}
    \item \textbf{Objetivo:} Prevenir recurrencias en pacientes que ya han tenido un evento cerebrovascular.
    \item \textbf{Acciones Clave:} Uso de fármacos antitrombóticos como antiagregantes o anticoagulantes (\cite{incn2018}).
\end{itemize}

En el contexto de la prevención de accidentes cerebrovasculares (ACV), la segmentación de imágenes es fundamental para evaluar la extensión del infarto y guiar decisiones clínicas importantes, como la administración de tratamientos y la identificación de pacientes que podrían beneficiarse de intervenciones adicionales.
\newline

Identificar áreas de riesgo en imágenes NCCT puede guiar la implementación de medidas preventivas secundarias, como la modificación de factores de riesgo (hipertensión, diabetes, etc.) y el seguimiento continuo para evitar futuros eventos cerebrovasculares.
\newline

Existe una superposición significativa entre los factores de riesgo que explican los eventos cerebrovasculares (\cite{sabihstroke}).  


\subsubsection {2.3.7 Tratamiento y Manejo}
\paragraph{Intervenciones: } Los pacientes con alto riesgo deben recibir tratamiento continuo para controlar la presión arterial, colesterol y diabetes. Estos tratamientos son de por vida y deben mantenerse incluso si los síntomas mejoran.


\paragraph{Recomendaciones}
\begin{itemize}
    \item \textbf{Actividad Física:} Se recomienda al menos 40 minutos de actividad física moderada a intensa, 3-4 días a la semana.
    \item \textbf{Dieta y Nutrición:} Reducir el sodio y aumentar el potasio, seguir una dieta mediterránea rica en vegetales, frutas y pescado.
    \item \textbf{Tratamiento de la Hipertensión y Dislipidemia:} Despistaje anual de presión arterial, uso de antihipertensivos para presiones $\geq 140/90$ mmHg, y estatinas si el LDL es $\geq 160$ mg/dL o en pacientes de alto riesgo.
\end{itemize}


La prevención primaria y secundaria de enfermedades cerebrovasculares es esencial para mejorar la salud pública. Un enfoque integral en la prevención, el control de factores de riesgo, y programas nacionales pueden reducir significativamente la morbi-mortalidad asociada. La detección precoz y la educación sobre estilos de vida saludables son clave para el manejo efectivo de estas enfermedades (\cite{incn2018}).
\begin{table}[h!]
    \centering
    \begin{tabular}{|p{0.48\textwidth}|p{0.48\textwidth}|}
        \hline
        \textbf{Factores de riesgo no modificables} & \textbf{Factores de riesgo modificables} \\
        \hline
        \begin{itemize}
            \item Edad (mayor riesgo a medida que aumenta la edad)
            \item Raza y origen étnico (mayor riesgo en africanos que en blancos)
            \item Género (mayor riesgo para los hombres, pero esta tendencia desaparece más allá de mediados de los 80)
            \item Antecedentes familiares (especialmente de enfermedad arterial cerebral)
            \item Factores de riesgo genéticos (por ejemplo, estados de hipercoagulabilidad, anemia de células falciformes, angiopatía amiloide cerebral)
        \end{itemize} &
        \begin{itemize}
            \item Hipertensión
            \item Diabetes mellitus
            \item Obesidad
            \item Perfil lipídico alterado
            \item Fibrilación auricular
            \item Enfermedad cardíaca estructural
            \item Estenosis de la arteria carótida
            \item Opciones de estilo de vida: dieta, ejercicio, tabaquismo, alcohol, etc.
        \end{itemize} \\
        \hline
    \end{tabular}
    \caption{Factores de riesgo para accidentes cerebrovasculares}
    \label{tab:risk_factors}
\end{table}


\subsubsection{2.3.8 Aprendizaje Profundo Multimodal}
En la actualidad, los sistemas de atención médica generan grandes cantidades de datos heterogéneos, como imágenes médicas, informes clínicos, señales fisiológicas y secuencias genómicas. La integración y análisis de estos datos multimodales es fundamental para mejorar el diagnóstico, la predicción de enfermedades y los tratamientos personalizados.
\newline

El aprendizaje profundo ha surgido como una de las principales técnicas para abordar la complejidad de los datos multimodales. Las arquitecturas de redes neuronales profundas, como las redes convolucionales (CNNs) y las redes recurrentes (RNNs), han mostrado un gran potencial para extraer características relevantes de distintos tipos de datos y combinarlas en modelos más robustos y precisos.
\newline

La fusión multimodal, que se refiere a la combinación de datos de diferentes modalidades en un solo modelo, ha demostrado ser efectiva en una variedad de aplicaciones médicas, incluyendo el diagnóstico basado en imágenes y texto, la predicción de resultados clínicos y el análisis de patrones complejos en datos fisiológicos.
\newline

Sin embargo, el análisis de datos multimodales también presenta desafíos importantes, como la integración de datos heterogéneos, la escasez de datos etiquetados y la necesidad de interpretar modelos complejos en contextos clínicos. Superar estos desafíos es clave para aprovechar el potencial del aprendizaje profundo en el ámbito médico.
\newline

\paragraph{Proceso de Aprendizaje Profundo Multimodal}

El proceso de aprendizaje profundo multimodal se centra en la integración de múltiples tipos de datos, como imágenes médicas y datos clínicos, para mejorar la precisión de las predicciones relacionadas con el riesgo de accidente cerebrovascular (ACV). Este proceso implica varios pasos clave:

  \begin{itemize}
    \item \textbf{Preprocesamiento de Datos:}
    \begin{itemize}
        \item Los datos de imágenes médicas y los datos clínicos se preprocesan para asegurar su calidad y relevancia. Esto incluye la normalización de las imágenes y la codificación de las características clínicas.
    \end{itemize}
    
    \item \textbf{Extracción de Características:}
    \begin{itemize}
        \item Las redes neuronales convolucionales (CNN) se utilizan para extraer características de las imágenes médicas, mientras que las redes neuronales recurrentes (RNN) o los modelos basados en atención se emplean para manejar los datos secuenciales o categóricos de los registros clínicos.
    \end{itemize}
    
    \item \textbf{Fusión de Modalidades:}
    \begin{itemize}
        \item Las características extraídas de cada modalidad se combinan utilizando métodos de fusión, ya sea en etapas tempranas (antes del aprendizaje conjunto) o en etapas tardías (después de que cada modalidad ha sido procesada de manera independiente). Este proceso de fusión es crucial para capturar las interacciones entre los datos de imágenes y los datos clínicos.
    \end{itemize}
    
    \item \textbf{Entrenamiento del Modelo:}
    \begin{itemize}
        \item El modelo multimodal se entrena utilizando un conjunto de datos etiquetado, optimizando sus parámetros mediante algoritmos como la retropropagación y el descenso de gradiente. Durante el entrenamiento, se utilizan técnicas de regularización para evitar el sobreajuste.
    \end{itemize}
    
    \item \textbf{Evaluación y Validación:}
    \begin{itemize}
        \item El rendimiento del modelo se evalúa utilizando métricas como la precisión, la sensibilidad, la especificidad, y el área bajo la curva ROC (AUC-ROC). Además, se emplean técnicas de validación cruzada para garantizar que el modelo generalice bien a nuevos datos.
    \end{itemize}
\end{itemize}

Este enfoque multimodal permite una integración más completa de la información disponible, lo que puede llevar a predicciones más precisas y robustas en el contexto clínico (\cite{BEHRAD2022117006}).
\newline

En el primer paso del análisis de datos médicos multimodales, se debe decidir las fuentes de datos, la estrategia de fusión, la estrategia de aprendizaje y la arquitectura de aprendizaje profundo (como se muestra en la Fig. 3). Elegir la combinación correcta de fuentes de datos en análisis multimodales es fundamental porque una combinación incorrecta conduce a un menor rendimiento. Las fuentes de datos deben proporcionar información complementaria para mejorar los resultados. El siguiente paso es decidir cómo integrar diferentes modalidades. Además, una adecuada debe elegir una estrategia de aprendizaje. Finalmente, los investigadores deberían elegir una arquitectura de red. Conocer diferentes arquitecturas de aprendizaje profundo ayuda a encontrar la arquitectura más adecuada para la investigación. En las siguientes secciones se explican estos conceptos (\cite{BEHRAD2022117006}).
\begin{figure}[!h]
	\centering
	\includegraphics[width=18cm]{PlanTesis_FormatoUNSA_2024/Imagenes/multimodal_dl_process.png}
    \caption{Cuatro decisiones sobre el análisis de datos médicos multimodales utilizando algoritmos de aprendizaje profundo.}
    \label{fig:estructura_cerebro}
\end{figure}

 %Inserta el capítulo 2

\renewcommand\refname{\large\textbf{Referencias}}
\bibliography{Referencias.bib}


\end{document}